\documentclass{article}

\usepackage[english]{babel}

\usepackage[letterpaper,top=2cm,bottom=2cm,left=3cm,right=3cm,marginparwidth=1.75cm]{geometry}

% Useful packages
\usepackage{amsmath}
\usepackage{graphicx}
\usepackage{amssymb}
\usepackage[colorlinks=true, allcolors=blue]{hyperref}

\title{Timofey Sedov}
\author{08.02.2004}
\date{}

\begin{document}
\maketitle

\section{Abstract}
This report represents the solution of
\href{https://moodle.innopolis.university/pluginfile.php/193734/mod_resource/content/2/GT1spr2
    4.pdf}{ The first practice assignment on Game Theory }. The aim of the
assignment is to implement a backward induction algorithm for the
\href{https://en.wikipedia.org/wiki/Dynamic_programming#egg_dropping_puzzle}{Dropping
    Bricks Puzzle}.

\section{Manual}
The source code of the solution is represented by a singe python script.
Assuming, you have pre-installed \verb|python3.11| the script can be ran by
executing the following command in the terminal:

\begin{verbatim}
    python TimofeySedov.py
\end{verbatim}

The cli-interface will give you further instructions on how to use the script.

Another options are:
\begin{itemize}
    \item \verb|python TimofeySedov.py --help| to get the list of available options
    \item \verb|python TimofeySedov.py --height HEIGHT| to play game with the explicitly specified height
    \item \verb|python TimofeySedov.py --test| to run the test suite which will test the solution using dynamic programming approach
    \item \verb|python TimofeySedov.py --test --height HEIGHT| to run the test suite with the specified height
\end{itemize}

\section{Implemented methods}
\subsection{Methods description}
The solution contains the following methods:
\begin{itemize}
    \item \verb|solve_min_drops_optimal| is a dynamic programming solution for dropping bricks puzzle deciding on the minimal number of drops given height and number of bricks
    \item \verb|solve_backward_induction| is a backward induction algorithm for the dropping bricks puzzle returning whether it is possible to find a mechanical strength of a brick for a given height, number of drops and number of bricks
    \item \verb|run_tests| is a function, which can test given function against dynamic programming solution, which as an arguments takes function returning whether there exists optimal strategy for a given height, number of drops and number of bricks, max height to test, drop range around the optimal height and number of bricks given at the start of the game
    \item \verb|play_game| is a function, which allows to play the game with the computer. Takes height as an argument
\end{itemize}

\subsection{Ideas behind the methods and the solution}
You can find the detailed explanation of the implemented methods in the source
code, but I am going to give a brief overview of the
\verb|solve_backward_induction| and \verb|play_game| methods.

\subsubsection{solve\_backward\_induction}
The method is using the 4d array which is representing the game state. After
defining the final positions for the player A as \verb|True|, the method
iterates over all possible not final positions and checks whether there exists
a move which in case of the opponent's optimal strategy still leads to the
final position. If such a move exists, the current position is also considered
as final. The method returns whether the initial position is final. And the 4d
array

\subsubsection{play\_game}
The method is using the \verb|solve_backward_induction| method to decide on the
optimal strategy for the computer. This method utilizes so called recovering
the solution approach to decide on the next move of the computer.

\section{Game Formal model}

\subsection{Finite Position Game Definition}

As mentioned in \href{
    https://moodle.innopolis.university/pluginfile.php/193734/mod_resource/content/2/GT1spr24.pdf#page=16}{
    Lecture Notes 1 page 16 } Finite Position Game for 2 players can be defined as
a tuple:
\[
    G = (P_A, P_B, M_A, M_B, F_A, F_B)
\]
Where:
\begin{itemize}
    \item $P_A$ and $P_B$ are disjoint finite sets of positions for players A and B
    \item $M_A$ and $M_B$ are admissible moves players A and B
    \item $F_A$ and $F_B$ are disjoint final positions
\end{itemize}

Let us proceed with the formal definition of the game in case of a
\href{https://moodle.innopolis.university/pluginfile.php/193734/mod_resource/content/2/GT1spr24.pdf#page=36}{Dropping
    Bricks Puzzle}

\subsection{Dropping Bricks Puzzle}

Let us assume, that the maximum height of the building is $H$ and maximum
number of drops is $D$. Now we can define positions for a players A and B as
follows:
\[
    P_A = (d, b, s, u) \;  P_B = (d, b, s, u, h)
\]
Where:
\begin{itemize}
    \item player A is a computer trying to find the mechanical strength of a brick and
          player B is a human

    \item $d$ is the number of drops left, $0 \leq d \leq D, d \in \mathbb{Z}$
    \item $b$ is the number of bricks left $0 \leq b \leq 2, b \in \mathbb{Z}$
    \item $s$ is the known safe height $1 \leq s \leq H, s \in \mathbb{Z}$
    \item $u$ is the known unsafe height $s < u \leq H + 1, u \in \mathbb{Z}$
    \item $h$ is the height of the drop $s < h < u, h \in \mathbb{Z}$
\end{itemize}

The initial position for the game is $(D, 2, 1, H + 1)$

Now we can define the set of admissible moves for players A and B as follows:
\begin{align*}
    M_A =  \{ ((d, b, s, u), ( & d - 1, b, s, u, h)) \; |                                                              \\
                               & | \; 1 \leq d \leq D, 1 \leq b \leq 2, 1 \leq s \leq H, s < u \leq H + 1, s < h < u\}
\end{align*}
and
\begin{align*}
    M_B = \{ ((d, b, s, u, h), (   & d, b, h, u)) \; |                                                                          \\
                                   & | \; 0 \leq d < D, 1 \leq b \leq 2, 1 \leq s \leq H, s < u \leq H + 1, s < h < u\} \; \cup \\
    \cup \; \{ ((d, b, s, u, h), ( & d, b - 1, s, h)) \; |                                                                      \\
                                   & | \; 0 \leq d < D, 1 \leq b \leq 2, 1 \leq s \leq H, s < u \leq H + 1, s < h < u\}
\end{align*}

Finally, we can define the set of final positions for players A and B as
follows:

\[
    F_A = \{ (d, b, s, u) \; | \; (d, b, s, u) \in P_A \text{ and } u - s = 1\}
\]
and
\[
    F_B = \{ (d, b, s, u) \in P_A \setminus F_A \; | \; d = 0 \text{ or } b = 0\}
\]
asdfsdf
\end{document}
